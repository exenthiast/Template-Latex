\documentclass[a4paper,oneside,11pt]{book}
\usepackage{graphicx}
\usepackage{titling}
\usepackage{listings}
\usepackage{geometry}
\usepackage{xcolor}
\usepackage{float}
\usepackage{hyperref}
\usepackage{indentfirst}
\title{Laporan Praktikum \\ Praktikum Basis Data\\
Pertemuan 2\\
Data Definition Language}
\author{Nama Mahasiswa: Alfiz Desta Ariant Permana(24/543220/SV/25173)}
\date{\today}

\geometry{
    left=3cm,
    right=2.5cm,
    top=3cm,
    bottom=3cm
}

\definecolor{codegreen}{rgb}{0,0.6,0}
\definecolor{codegray}{rgb}{0.5,0.5,0.5}
\definecolor{codepurple}{rgb}{0.58,0,0.82}
\definecolor{backcolour}{rgb}{0.95,0.95,0.92}

\lstdefinestyle{sqlstyle}{
    backgroundcolor=\color{backcolour},   
    commentstyle=\color{codegreen},
    keywordstyle=\color{codepurple},
    numberstyle=\tiny\color{codegray},
    stringstyle=\color{codegreen},
    basicstyle=\ttfamily\footnotesize,
    breakatwhitespace=false,         
    breaklines=true,                 
    captionpos=b,                    
    keepspaces=true,                 
    numbers=left,                    
    numbersep=5pt,                  
    showspaces=false,                
    showstringspaces=false,
    showtabs=false,                  
    tabsize=2,
    frame=single
}

\lstdefinestyle{htmlstyle}{
    backgroundcolor=\color{backcolour},   
    commentstyle=\color{codegreen},
    keywordstyle=\color{codepurple},
    numberstyle=\tiny\color{codegray},
    stringstyle=\color{codegreen},
    basicstyle=\ttfamily\footnotesize,
    breakatwhitespace=false,         
    breaklines=true,                 
    captionpos=b,                    
    keepspaces=true,                 
    numbers=left,                    
    numbersep=5pt,                  
    showspaces=false,                
    showstringspaces=false,
    showtabs=false,                  
    tabsize=2,
    frame=single
}



\begin{document}
\begin{titlingpage} %This starts the title page
\begin{center}

\vspace{4cm} %You can control the vertical distance
\begin{huge} 
\textbf{\thetitle} \\
\end{huge}
\vspace{2cm}

\includegraphics[height=8cm]{lambang ugm.png}\\ %Put the logo you want here
\begin{Large}
\vspace{4cm} %Put the distance you need.
\theauthor\\
Kelas A2 \\ %The name your university
\end{Large}
\begin{large}
Dosen Pengampu: Dinar Nugroho Pratomo, S.Kom., M.IM., M.Cs.\\
\end{large}


\thedate
\end{center}
\end{titlingpage}


\tableofcontents

\chapter{Tujuan Praktikum} 
\begin{enumerate}
  \item 
\end{enumerate}

\chapter{Dasar Teori}


\chapter{Hasil dan Pembahasan}
\begin{lstlisting}[language=SQL, style=sqlstyle]
    CREATE TABLE pembelian (
      kode_pembelian char(6) NOT NULL PRIMARY KEY,
      kode_barang char(6) NOT NULL,
      kode_customer char(6) NOT NULL,
      jumlah_pasok INT,
      
      FOREIGN KEY (kode_barang) REFERENCES barang(kode_barang),
      FOREIGN KEY (kode_customer) REFERENCES customer(kode_customer)
     );
    \end{lstlisting}

\begin{lstlisting}[language=HTML, style=htmlstyle]
    <!DOCTYPE html>
    <html>
    <head>
        <title>Judul Halaman</title>
    </head>
    <body>
        <h1>Ini adalah heading</h1>
        <p>Ini adalah paragraf.</p>
    </body>
    </html>
\end{lstlisting}


\chapter{Kesimpulan}

\chapter{Daftar Pustaka}
\begin{enumerate}
    \item itboxddl, \url{https://itbox.id/blog/ddl-adalah/}
    \item datacampCommandsDefinitive, \url{https://www.datacamp.com/tutorial/sql-ddl-commands}
    \item telkomuniversitySimakPengertian, \url{https://it.telkomuniversity.ac.id/perintah-dasal-sql/}
\end{enumerate}



\end{document}
